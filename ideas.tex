\documentclass[a4paper,10pt]{article}

\usepackage[spanish]{babel}
\usepackage[utf8]{inputenc}

\begin{document}
\section*{Reconocimiento automático del habla en Lojban}
	Se pretende traducir un archivo de audio que contenga una
	frase en lojban, a su representación escrita. Para poder realizar
	dicha representación se requiere la identificación de sílabas,
	su correcta agrupación en palabras y su traducción escrita.

	La elección del lenguaje se debe a que el mismo es de escritura
	y pronunciación fonética, de gramática no ambigua y de
	diccionario reducido. Lo cual debería de simplificar el
	procesamiento, haciéndolo más robusto al ruido y posible de
	realizar online.

	Para simplificar el problema se considerará solo un hablante
	por vez y se limitará al reconocimiento de números.

	Un uso podría ser para que personas paralíticas puedan escribir
	en computadora con la voz, ya que como se comentó anteriormente
	es un lenguaje menos sensible al ruido.

	\subsection*{Referencias}
		\begin{itemize}
			\item Fundamentals of Speech Recognition (by:
			Lawrence Rabiner, Biing-Hwang Juang)
			\item Introduction to Digital Speech Processing
			(by: Lawrence Rabiner and Ronald Schafer)
			\item Digital Speech Processing, Synthesis,
			and Recognition (by: Sadaoki Furui)
			\item Speech Recognition with Primarily Temporal
			Cues (by: Robert V. Shannon(1), Fan-Gang Zeng,
			Vivek Kamath, John Wygonski, Michael Ekelid)
			\item "Representación de la voz en el
			reconocimiento del habla." Quark 21 (2001):
			63-71. (by: Nadeu, Climent)
		\end{itemize}
	
	\clearpage

\section*{Detección de la partitura de una canción instrumental}
	La idea del proyecto consiste en generar la partitura de una
	canción instrumental ingresada por el usuario.  El algoritmo
	consistirá en un análisis en tiempo y frecuencia  de la
	señal con el fin de ir detectando cada una de las notas que se
	interpretan a lo largo de la canción.

	Para simplificar el problema, realizaremos el trabajo a partir
	de una canción interpretada por un solo instrumento.

	Consideramos que es una aplicación muy interesante para personas
	que están aprendiendo a tocar instrumentos y todavía no poseen
	un oído musical muy desarrollado como para poder conocer la
	partitura de una canción mientras la escuchan.

	\subsection*{Referencias}
		\begin{itemize}
			\item Recognition of piano notes with the aid
			of FRM filters (Say Wei Foo; Wei Thai Lee).
			\item Extracting note onsets from musical
			recordings (Alonso, M.; Richard, G. , David, B.).
			\item Application of FRM filters for musical
			notes separation (Say Wei Foo; Lee, E.W.T.L.).
			\item Musical note segmentation employing
			combined time and frequency analyses (Velikic,
			G.; Titlebaum, Edward L. ; Bocko, M.F.).
		\end{itemize}

	\clearpage

\section*{Ecualizador Paramétrico}
	La idea del proyecto es mejorar subjetivamente la calidad de
	la señal de audio o realzar el aporte de un instrumento en
	particular.

	Para esto se trabajará en el espectro de la señal, utilizando un
	banco de filtros donde se prodrá ajustar la ganancia, frecuencia
	central y ancho de banda individualmente.

	Además se contemplarán operaciones de expansión, compresión
	y desplazamiento de las frecuencias en una banda

	Al resultado final se lo comparará de forma subjetiva con la
	señal original

	\subsection*{Referencias}
		\begin{itemize}
			\item Control activo de ruido con ecualización
			del camino secundario. ( by: A. Ortega,
			E. Masgrau, E. Lleida).
			\item Aplicación De Un Procesador Difuso Tipo
			II En La Ecualización De Canales No Lineales
			Y Variantes En El Tiempo. ( by: Miguel A.
			Melgarejo Rey, Carlos A. Peña Reyes).
			\item Algoritma de ecualización para un receptor
			DRM. (by: E. Alcalde, G.  Agujeta, J. Prieto,
			I. Izquierdo, A. Sanzberro).
		\end{itemize}

	\clearpage
\end{document}


