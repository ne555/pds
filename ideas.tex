\documentclass[a4paper,10pt]{article}

\usepackage[spanish]{babel}
\usepackage[utf8]{inputenc}

\begin{document}
\section*{Reconocimiento automático del habla en Lojban}
	Se pretende traducir un archivo de audio que contenga una frase en lojban, a su representación escrita.
	Se requiere la identificación de sílabas y su correcta agrupación en palabras. 

	La elección del lenguaje se debe a que es de escritura y pronunciación fonética, de gramática no ambigua y de diccionario reducido.
	Lo cual debería de simplificar el procesamiento, haciéndolo más robusto al ruido y posible de realizar online

	Se considerará solo un hablante por vez. 

	Como acotación se podrá limitar el problema al reconocimiento de números.

	\subsection*{Referencias}
		\begin{itemize}
			\item Fundamentals of Speech Recognition (by: Lawrence Rabiner, Biing-Hwang Juang)
			\item Introduction to Digital Speech Processing (by: Lawrence Rabiner and Ronald Schafer)
			\item Digital Speech Processing, Synthesis, and Recognition (by: Sadaoki Furui)
			\item Speech Recognition with Primarily Temporal Cues (by: Robert V. Shannon(1), Fan-Gang Zeng, Vivek Kamath, John Wygonski, Michael Ekelid)
			\item "Representación de la voz en el reconocimiento del habla." Quark 21 (2001): 63-71. (by: Nadeu, Climent)
		\end{itemize}

\end{document}


