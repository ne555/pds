\documentclass[a4paper]{article}
%\documentclass[conference,a4paper,10pt,oneside,final]{tfmpd}

\usepackage[spanish]{babel}
\usepackage[utf8]{inputenc}

\usepackage{gnuplot-lua-tikz}
\usepackage{tikz}
\usetikzlibrary{positioning,calc}

\title{Reconocimiento automático de dígitos en Lojban a partir de señal sonora}
\author{Bedrij \and Goette \and Gómez}
\date{}
%esto es un comentario

\begin{document}
	\maketitle
	\section{Estado del arte}
		A partir de una búsqueda bibliográfica hemos encontrado que existen
		diversos métodos que se pueden utilizar para realizar un análisis de
		las señales de audio con el fin de extraer características de las mismas
		que puedan utilizarse para poder obtener sus representaciones escritas.

		Entre dichos métodos se destacan la utilización de coeficientes
		cepstrales [1], codificación predictiva lineal (LPC ) [2], perceptual
		linear predictive (PLP) [3], pares de líneas espectrales (LSP) [4][5],
		banco de filtros [6].


		Asimismo existen numerosas técnicas que permiten clasificar las señales
		mediante las características previamente obtenidas que incluyen desde
		cálculos de distancias entre la señal que se desea procesar y una base
		de datos de ejemplos a métodos de inteligencia computacional. Entre
		éstas podemos nombrar el cálculo de la distancia euclídea, Dynamic
		Time Warping [7],  cuantización vectorial [8], perceptrón multicapa
		[9], etcétera.

	\section{Introducción}
		Se pretende traducir un archivo de audio con números en Lojban
		a su representación escrita.
		%justificación del lenguaje
		Si bien no pueden aprovecharse el uso de la gramática
		no ambigua y libre de contexto (que fue lo que impulsó la
		elección del idioma) se aprecian algunas características que
		pueden facilitar la tarea, como ser
		\begin{itemize}\itemsep0pt
		\item los números están compuestos por concatenación de dígitos
		\item todos los dígitos son monosílabos compuestos por una consonante y vocal
			(en ese orden)
		\item se utiliza la misma vocal en a lo sumo tres dígitos,
		\item las consonantes que acompañan vocales repetidas son de sonidos muy diferentes
		\end{itemize}

		En la Cuadro~\ref{tab:digitos} puede verse la correspondencia entre los dígitos decimales
		y su representación en Lojban. La pronunciación es igual al español
		con excepción de los números $3$ y $6$ que se pronuncian como indica
		entre paréntesis.

		%lista de dígitos
		\begin{table}[h]
			\center
			\begin{tabular}{lllll}
				$0$ no & $1$ pa & $2$ re & $3$ ci(shi) & $4$ vo\\
				$5$ mu & $6$ xa(ja) & $7$ ze & $8$ bi & $9$ so\\
			\end{tabular}
			\caption{Dígitos en lojban}
			\label{tab:digitos}
		\end{table}

		%Procesamiento
		Se procesará cada dígito de forma aislada, suponiendo
		resuelto el problema de separar un número en sus dígitos.
		La clasificación se logra mediante una comparación contra una
		base de datos buscándose el más parecido.

		Por motivos de costo computacional, y de robustez, no se
		utilizará la señal temporal de forma cruda sino que se
		extraerán características.  Además, considerando las
		diferencias de tiempo y velocidad de habla, se utilizará
		\emph{dynamic time warping} (DTW) como medida de distancia.

	\section{Características}
		Para distinguir los fonemas se analizará la
		respuesta en frecuencia de las vocales y consonantes.
		Esta información se encontrará condensada en diversos
		parámetros %lista
			\begin{itemize} \itemsep0pt
				\item Momentos estadísticos (media, varianza, simetría, curtosis)
				\item Coeficientes ceptrales (formantes 1 y 2)
				\item Modelo AR equivalente. Representado mediante PLP o LSP
			\end{itemize}

		Estos parámetros serán calculados por ventanas
		temporales de $20ms$ a las que se ajustó mediante la
		escala de Mel.

		El diccionario se constituirá por la concatenación de
		estas características a lo largo de la vocal y del fonema


	\section{Clasificación}
		Como se mencionó antes las vocales se encuentran
		distribuidas en los dígitos, no hallándose más de
		una por dígito y no teniendo más de tres dígitos la
		misma vocal.

		Esto sugiere que la identificación de la vocal podría
		reducir el problema considerablemente.	Problema un poco
		más sencillo, posible de resolverse simplemente mediante
		el cálculo de las primeras formantes.	Sin embargo surge
		el inconveniente de separar la vocal de la consonante.
		Supondremos que el uso de una heurística para distinguir
		una porción estacionaria será suficiente.

		Luego de identificar la vocal, se tendrán a lo sumo
		tres candidatos.  Se resolverá mediante el uso de DTW
		durante todo el fonema

	\section{Validación}
		El resultado de la DTW nos brinda una medida de confianza
		en la clasificación.  Para probar la robustez de los
		algoritmos, los casos de prueba se contaminarán con
		distintos niveles de ruido, observando el porcentaje de
		aciertos en cada caso


\newpage
		\begin{figure}
			\fbox{
				\begin{tikzpicture}[
	remember picture,
	nonterminal/.style={
		rectangle,
		minimum size=6mm,
		%very thick,
		solid,
		inner sep=1mm,
		draw
	},
	terminal/.style={
		circle,
		solid,
		minimum size=3mm,
		inner sep=0pt,
		%very thick,
		draw
	},
	txt/.style={
		nonterminal,
		text width=2cm,
		anchor=center,
		align=center,
	},
	group/.style={
		draw,
		line width=1pt,
		dash pattern=on 3pt off 1pt,
		inner sep=4mm,
		rectangle,
		%rounded corners
	},
]
\matrix[row sep=5mm, column sep=5mm]{
	%first row
	%& \node (window) [nonterminal] {Ventana}; & \\
	%second row
	\node (in) [terminal] {}; & 
	\node (x) [terminal] {$\times$}; & & &
	%\node (mel) [nonterminal] {MEL}; &
	\node (caracteristicas) [group] {
		\begin{tikzpicture}
			\node (AR) [txt] {Modelo AR};
			\node (CC) [txt,below of=AR] {Coeficientes Ceptrales};
			\node (EM) [txt,below of=CC] {Momentos Estadísticos};
		\end{tikzpicture}
	}; &
	\node (clasificador) [group] {
		\begin{tikzpicture}
			\node (DTW) [txt] {DTW};
			\node (dict) [txt,below of=DTW] {Diccionario};
			\draw[solid,-latex] (dict) -> (DTW);
		\end{tikzpicture}
	}; \\
};
\node (window) [node distance=4em,above of=x,nonterminal] {
	\begin{tikzpicture}[gnuplot,
	xscale=0.25,
	yscale=0.15
]
%% generated with GNUPLOT 4.6p2 (Lua 5.2; terminal rev. 99, script rev. 100)
%% Tue 28 May 2013 01:31:22 AM ART
\path (0.000,0.000) rectangle (8.000,6.000);
\gpcolor{rgb color={0.000,0.000,0.000}}
\gpsetlinetype{gp lt plot 0}
\gpsetlinewidth{0.50}
\draw[gp path] (0.460,0.368)--(0.515,0.371)--(0.570,0.380)--(0.625,0.394)--(0.680,0.415)%
  --(0.735,0.441)--(0.790,0.473)--(0.845,0.510)--(0.900,0.553)--(0.955,0.601)--(1.010,0.655)%
  --(1.065,0.714)--(1.120,0.777)--(1.175,0.846)--(1.230,0.919)--(1.285,0.997)--(1.340,1.079)%
  --(1.395,1.165)--(1.450,1.256)--(1.505,1.349)--(1.560,1.447)--(1.615,1.547)--(1.670,1.651)%
  --(1.725,1.757)--(1.780,1.865)--(1.835,1.976)--(1.890,2.089)--(1.945,2.203)--(2.000,2.319)%
  --(2.055,2.436)--(2.110,2.553)--(2.165,2.672)--(2.221,2.790)--(2.276,2.908)--(2.331,3.026)%
  --(2.386,3.143)--(2.441,3.260)--(2.496,3.375)--(2.551,3.488)--(2.606,3.600)--(2.661,3.710)%
  --(2.716,3.817)--(2.771,3.922)--(2.826,4.024)--(2.881,4.123)--(2.936,4.219)--(2.991,4.311)%
  --(3.046,4.399)--(3.101,4.483)--(3.156,4.563)--(3.211,4.639)--(3.266,4.709)--(3.321,4.776)%
  --(3.376,4.837)--(3.431,4.893)--(3.486,4.944)--(3.541,4.990)--(3.596,5.030)--(3.651,5.065)%
  --(3.706,5.094)--(3.761,5.117)--(3.816,5.134)--(3.871,5.146)--(3.926,5.152)--(3.981,5.152)%
  --(4.036,5.146)--(4.091,5.134)--(4.146,5.117)--(4.201,5.094)--(4.256,5.065)--(4.311,5.030)%
  --(4.366,4.990)--(4.421,4.944)--(4.476,4.893)--(4.531,4.837)--(4.586,4.776)--(4.641,4.709)%
  --(4.696,4.639)--(4.751,4.563)--(4.806,4.483)--(4.861,4.399)--(4.916,4.311)--(4.971,4.219)%
  --(5.026,4.123)--(5.081,4.024)--(5.136,3.922)--(5.191,3.817)--(5.246,3.710)--(5.301,3.600)%
  --(5.356,3.488)--(5.411,3.375)--(5.466,3.260)--(5.521,3.143)--(5.576,3.026)--(5.631,2.908)%
  --(5.686,2.790)--(5.742,2.672)--(5.797,2.553)--(5.852,2.436)--(5.907,2.319)--(5.962,2.203)%
  --(6.017,2.089)--(6.072,1.976)--(6.127,1.865)--(6.182,1.757)--(6.237,1.651)--(6.292,1.547)%
  --(6.347,1.447)--(6.402,1.349)--(6.457,1.256)--(6.512,1.165)--(6.567,1.079)--(6.622,0.997)%
  --(6.677,0.919)--(6.732,0.846)--(6.787,0.777)--(6.842,0.714)--(6.897,0.655)--(6.952,0.601)%
  --(7.007,0.553)--(7.062,0.510)--(7.117,0.473)--(7.172,0.441)--(7.227,0.415)--(7.282,0.394)%
  --(7.337,0.380)--(7.392,0.371)--(7.447,0.368);
%% coordinates of the plot area
\gpdefrectangularnode{gp plot 1}{\pgfpoint{0.460cm}{0.368cm}}{\pgfpoint{7.447cm}{5.631cm}}
\end{tikzpicture}
%% gnuplot variables

}; 
\node (out) [right=of DTW,terminal] {};

\draw[-latex] (in) -> (x);
%\draw[-latex] (x) -> (mel);
%\draw[-latex] (mel) -> (caracteristicas);
\draw[-latex] (x) -> (caracteristicas);
%\draw[-latex] (caracteristicas.east) -- (DTW.west);
\draw[-latex] (caracteristicas.east|-DTW.west) -> (DTW.west);
%\draw[-latex] (caracteristicas) -> (DTW);
\draw[-latex] (DTW.east) -> (out);
\draw[-latex] (window) -> (x);

\node at (window.north) [above] {Ventana};
\node at (caracteristicas.south) [below] {Características};
\node at (clasificador.south) [below] {Clasificador};
\node at (in.west) [left] {Input};
\node at (out.east) [right] {Output};
\end{tikzpicture}


			}
			\caption{Diagrama en bloques}
		\end{figure}
	\section{Bibliografía}
		[1] Ecualización de Histogramas Adaptativa en
		el Dominio Cepstral para Reconocimiento de Voz Robusto
		\emph{Carmen Benítez, Ángel de la Torre, José C. Segura, Javier Ramírez, Antonio J. Rubio.}


		[2] Feature Extraction for Speech Recogniton
		\emph{Manish P. Kesarkar}


		[3] Perceptual Linear Predictive (PLP) Analysis of Speech
		\emph{Hynek Hermansky}


		[4] A Review of Line Spectral Pairs
		\emph{Ian Vince McLoughlin}


		[5] The Distance Measure for line Spectrum Pairs
		Applied to Speech Recognition
		\emph{Fang Zheng, Zhanjiang Song, Ling Li, Wenjian Yu, Fengzhou Zheng, and Wenhu Wu}


		[6] Digital Processing of Speech Signals
		\emph{Rabiner Schafer}


		[7] Dynamic Time Warping Algorithm Review.
		\emph{Pavel Senin}


		[8] Fundamentals of Speech Recognition.
		\emph{Lawrence Rabiner, Biing-Hwang Juang}

		[9] Continuous Speech Recognition using PLP Analysis with Multilayer Perceptrons.
		\emph{Morgan, N.; Hermansky, H.; Bourlard, H.; Kohn, P.; Wooters, C.}
		
	

\end{document}

