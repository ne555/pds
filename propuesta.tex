\documentclass[a4paper]{article}

\usepackage[spanish]{babel}
\usepackage[utf8]{inputenc}

\title{Reconocimiento automático de dígitos en Lojban a partir de señal sonora}
\author{Bedrij \and Goethe \and Gómez}
\date{}
%esto es un comentario

\begin{document}
	\maketitle
	\section{Introducción}
	Se pretende traducir un archivo de audio que números en Lojban a su representación escrita.
	%justificación del lenguaje
	Si bien no pueden aprovecharse el uso de la gramática no ambigua y libre de contexto
	(que fue lo que impulsó la elección del idioma)
	se aprecian algunas características que pueden facilitar la tarea, como ser
	que los números están compuestos por concatenación de dígitos
	que todos los dígitos son monosílabos compuestos por una consonante y vocal (en ese orden)
	que se utiliza la misma vocal en a lo sumo tres dígitos, 
	que las consonantes que acompañan vocales repetidas son de sonidos muy diferentes

	%lista de dígitos
	0 no pa re ci(shi) vo
	mu xa(ja) ze bi so 9

	%Procesamiento
	Se procesará cada dígito de forma aislada, suponiendo resuelto el problema de separar un número en sus dígitos.
	La clasificación se logra mediante una comparación contra una base de datos buscándose el más parecido.
	
	Por motivos de costo computacional, y de robustez, no se utilizará la señal temporal de forma cruda sino que se extraerán características.
	Además, considerando las diferencias de tiempo y velocidad de habla, se utilizará \emph{dynamic time warping} (DTW) como medida de distancia.

	\section{Características}
		Para distinguir los fonemas se analizará la respuesta en frecuencia de las vocales y consonantes.
		Esta información se encontrará condensada en diversos parámetros
		%lista
			- Momentos estadísticos (media, varianza, simetría, curtosis)
			- Coeficientes ceptrales (formantes 1 y 2)
			- Modelo AR equivalente. Representado mediante PLP o LSP

		Estos parámetros serán calculados por ventanas temporales de $20ms$ a las que se ajustó mediante la escala de Mel.

		El diccionario se constituirá por la concatenación de estas características a lo largo de la vocal y del fonema

	\section{Clasificación}
		Como se mencionó antes las vocales se encuentran distribuídas en los dígitos, no hallándose más de una por dígito
		y no teniendo más de tres dígitos la misma vocal.

		Esto sugiere que la identificación de la vocal podría reducir el problema considerablemente. 
		Problema un poco más sencillo, posible de resolverse simplemente mediante el cálculo de las primeras formantes.
		Sin embargo surge el inconveniente de separar la vocal de la consonante. 
		Supondremos que el uso de una heurística para distinguir la porción de estacionariedad será suficiente.

		Luego de identificar la vocal, se tendrán a lo sumo tres candidatos. 
		Se resolverá mediante el uso de DTW durante todo el fonema

		
	

\end{document}

