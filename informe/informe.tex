\documentclass[conference,a4paper,10pt,oneside,final]{tfmpd}
\usepackage[utf8]{inputenc}   % caracteres especiales (acentos, eñes)
\usepackage[spanish]{babel}     % varias definiciones para el español
\usepackage{graphicx}           % inserción de graficos
\usepackage{tikz}

\usepackage{gnuplot-lua-tikz}
\usepackage{tikz}
\usetikzlibrary{positioning,calc}

\begin{document}
\title{Reconocimiento automático de dígitos en Lojban a partir de señal sonora}

\author{Walter O. Bedrij,
        Marcelo Goette y 
        Virginia M. Gómez \\
\textit{Trabajo práctico final de ``Muestreo y procesamiento digital'', II-FICH-UNL.}}

\markboth{MUESTREO Y PROCESAMIENTO DIGITAL: TRABAJO FINAL}{}


\maketitle
\begin{abstract}
	Uno de los futuros objetivos del lenguaje lojban es mejorar la
	comunicación humano--computadora.  El presente trabajo aborda
	la traducción de archivos de audio a su representación escrita
	para luego ser interpretados por una computadora, acotado al
	reconocimiento de números.


	Se dividirá el problema de identificación en 2 partes para
	distinguir cada fonema.  Primeramente, identificaremos la vocal
	del este de cada número utilizando line spectral pairs (LSP).
	Luego de identificar las vocales solo quedaría determinar la
	consonante para saber el fonema. 
	Esto se hará mediante el calculo de los momentos estadísticos en el espectro,
	como son de longitud variable aplicamos el algoritmo dynamic time warping (DTW)
	utilizando su costo como medida de distancia.


	Si bien los resultados de clasificación fueron aceptables,
	el método resulta muy vulnerable al ruido

\end{abstract}


\section{Introducción}
	El objetivo del trabajo es realizar un algoritmo que permita la identificación de dígitos
	en un archivo de audio.

	No se contempla concatenaciones u otros sonidos que no sean los correspondientes a los dígitos

	Si bien no pueden aprovecharse el uso de la gramática
	no ambigua y libre de contexto (que fue lo que impulsó la
	elección del idioma) se aprecian algunas características que
	pueden facilitar la tarea, como ser
	\begin{itemize}\itemsep0pt
	\item los números están compuestos por concatenación de dígitos
	\item todos los dígitos son monosílabos compuestos por una consonante y vocal
		(en ese orden)
	\item se utiliza la misma vocal en a lo sumo tres dígitos,
	\item las consonantes que acompañan vocales repetidas son de sonidos muy diferentes
	\end{itemize}

	En la Tabla~\ref{tab:digitos} puede verse la correspondencia entre los dígitos decimales
	y su representación en Lojban. La pronunciación es igual al español
	con excepción de los números $3$ y $6$ que se pronuncian como indica
	entre paréntesis.

	%lista de dígitos
	\begin{table}[h]
		\center
		\begin{tabular}{lllll}
			$0$ no & $1$ pa & $2$ re & $3$ ci(shi) & $4$ vo\\
			$5$ mu & $6$ xa(ja) & $7$ ze & $8$ bi & $9$ so\\
		\end{tabular}
		\caption{Dígitos en lojban}
		\label{tab:digitos}
	\end{table}


   	 El procesamiento es como sigue:
   	 \begin{enumerate}
   		 \item Eliminación del silencio
   		 \item Separación de las partes correspondientes a `vocal' y `consonante'
   		 \item Identificación de vocal
   		 \item Identificación de la consonante, teniendo en cuenta la vocal
   	 \end{enumerate}
	 %diagrama en bloques
	 \begin{figure}
		\resizebox{0.5\textwidth}{!}{
			\input{diagrama}
		}
		\caption{Diagrama en bloques}
	 \end{figure}
	

\section{Materiales y métodos}
	\subsection{Base de datos}
		Los archivos de audio fueron generados por una sola persona que recitó los dígitos en orden.
		Se realizaron $20$ grabaciones, dejando un tiempo prudencial entre cada dígito.

		Los dígitos se separaron considerando una caída de $26dB$ en la amplitud, durante al menos
		$0.5s$ como silencio. Cada dígito se dejó con $100ms$ al inicio y al final de `silencio'

		De las $20$ realizaciones, $10$ fueron destinadas a prueba y $10$ a entrenamiento.


	\subsection{Identificación de la vocal}
		La señal se dividerá en ventanas de $\approx 20ms$, considerándose vocal si 
		su energía supera un umbral. 
		Además se observó que el $lsp(2)$ puede ser usado para discernir sonidos sordos (valores altos) 
		de sonoros

		Como característica clasificadora se utilizarán los \emph{line spectral pairs} (LSP).
		Los LSP son una representación artificial de la respuesta del tracto vocal $H(z)$,
		usando como base los coeficientes del sistema AR equivalente
		\begin{eqnarray*}
			A(z) &=& 1/H(z) \\
			P(z) &=& A(z) + z^{-(p+1)}A(z^{-1}) \\
			Q(z) &=& A(z) - z^{-(p+1)}A(z^{-1})
		\end{eqnarray*}
		siendo los LSP las raíces de los polinomios $P$ y $Q$

		De los LSP nos interesa simplemente el ángulo, 
		ya que se relaciona con la frecuencia de las formantes 
		(como se observa en la fig~\ref{fig:tracto}).
		\begin{figure}
			\includegraphics[width=0.5\textwidth]{imagen/3.png}
			\caption{
				Respuesta al tracto vocal de `ci' (porción central de la vocal)
				Se observa como las raíces de $P(z)$ y $Q(z)$ representan cercanamente a las formantes
			}
			\label{fig:tracto}
		\end{figure}

		\subsubsection{Generación de los patrones}
		Los patrones se crean mediante un promedio ponderado de los LSP
		a lo largo del tiempo, asignándole menor peso a las primeras muestras
		con el fin de evitar perturbaciones producidas por la consonante.
		Luego, se promedian todas las realizaciones de las muestras de entrenamiento.

		No se utilizan todos los coeficientes,
		ya que engloban más información de la necesaria.
		Empiricamente se determinó que se obtienen mejores resultados 
		al usar los coeficientes $3$ al $10$

		Para dar igual importancia a cada coeficiente, se los dividirá por la norma infinito
		a lo largo de los dígitos.
			
		\subsubsection{Clasificación}
		Para identificar la vocal, se divide en ventanas de $\approx 20ms$ y se comparan los
		LSP de cada ventana contra el de los patrones.
		El ganador se decide por votación.


		Para considerar solamente la porción estable de la vocal,
		del total de ventanas se descartan el primer y último cuarto
		%el método presenta problemas en 'no' 'vo' 'mu'


		El patrón más cercano (según la distancia euclídea) a los coeficientes
		calculados para una ventana, se considera el ganador de la ventana.
		La vocal ganadora es aquella con mayor cantidad de patrones ganadores.

	\subsection{Identificación de la consonante}
	\subsubsection{Patrones para detección de consonantes}

		La base de datos para detectar la consonante está constituida por una
		matriz por cada una de las señales utilizadas como patrón las cuales
		contienen características estadísticas del espectro de frecuencia en
		ventanas tomadas a lo largo del tiempo de la señal.

		Para la  generación de cada matriz realizamos un ventaneo, mediante
		ventanas de Hanning de tamaño fijo (aproximadamente $10 ms$) a lo largo
		de toda la señal y luego descartamos aquellas en las que su energía no
		superaba un determinado umbral ya que éstas eran consideradas silencio.

		Luego, de todas las ventanas obtenidas tomamos solo las que correspondían
		al primer cuarto de la señal porque consideramos que en ese sector
		estaría contenida la información de la consonante.

		A continuación, aplicamos la transformada rápida de Fourier a cada
		ventana y calculamos la media, desvío estándar, asimetría y curtosis
		de cada una mediante la función statistics de octave.

		De esta manera obtuvimos una matriz de $4 \times n$ elementos donde cada fila
		contiene la información de cada medida estadística y la cantidad $n$
		de columnas está determinada por el número de ventanas que contendrían
		la información de la consonante que es variable en cada señal.

		
		\begin{figure}
			\includegraphics[width=0.5\textwidth]{imagen/caracteristicas.png}
			\caption{
				Desvío ($*$) y asimetría ($\circ$) a lo largo del tiempo
				en el sector correspondiente a la consonante
			}
			\label{fig:tracto}
		\end{figure}


	\subsubsection{Procedimiento para determinar la consonante}

		El procedimiento aplicado para detectar la consonante en cada dígito que
		queremos identificar consiste en primer lugar en obtener la matriz con
		las medidas estadísticas del espectro de frecuencia de la misma forma
		que generamos las matrices de las señales patrones. En segundo lugar
		aplicamos el algoritmo Dynamic Time Warping  para lograr que la matriz
		obtenida de esta señal y cada una de las de la base de datos tengan la
		misma dimensión  y utilizamos el costo que obtuvo este algoritmo para
		identificar la consonante de la señal analizada.

		Como dijimos en la introducción, los números en el lenguaje Lojban
		están conformados por una consonante y una vocal, por lo que en
		la mayoría de los casos con saber la vocal quedan solo $2$ dígitos
		candidatos. Por lo tanto luego de haber extraído la vocal en el paso
		anterior ahora realizamos la evaluación para la consonante en los $5$
		patrones de cada uno de los dígitos que comparten la vocal obtenida por
		lo que en vez de tener que hacer $50$ evaluaciones, hacemos $10$ para los
		dígitos cuyas vocales son `a', `e', `i'; $15$ para `o' y en el caso del dígito
		cuya vocal es `u' ya queda determinado porque es el único que contiene
		dicha vocal.


\section{Resultados}
	\begin{table}
	\[
		\begin{array}{cccccccccc}
10 & 0 & 0 & 0 & 0 & 0 & 0 & 0 & 0 & 0 \\
0 & 10 & 0 & 0 & 0 & 0 & 0 & 0 & 0 & 0 \\
0 & 0 & 8 & 0 & 0 & 0 & 0 & 1 & 0 & 1 \\
0 & 0 & 0 & 8 & 0 & 0 & 0 & 1 & 1 & 0 \\
0 & 0 & 0 & 0 & 7 & 2 & 0 & 0 & 0 & 1 \\
0 & 0 & 0 & 0 & 0 & 10 & 0 & 0 & 0 & 0 \\
0 & 3 & 0 & 0 & 0 & 0 & 7 & 0 & 0 & 0 \\
0 & 0 & 0 & 0 & 0 & 0 & 0 & 10 & 0 & 0 \\
0 & 0 & 0 & 0 & 0 & 0 & 0 & 0 & 10 & 0 \\
0 & 0 & 0 & 0 & 0 & 0 & 0 & 0 & 0 & 10 \\
		\end{array}
	\]
		\caption{Matriz de confusión para una $SNR > 40dB$.\\
		Porcentaje de aciertos de $90\%$}
	\end{table}

	\begin{table}
	\[
		\begin{array}{cccccccccc}
0 & 0 & 0 & 0 & 0 & 10 & 0 & 0 & 0 & 0 \\
0 & 0 & 0 & 0 & 1 & 5 & 0 & 0 & 0 & 4 \\
1 & 0 & 0 & 0 & 0 & 9 & 0 & 0 & 0 & 0 \\
0 & 0 & 0 & 9 & 0 & 0 & 0 & 0 & 1 & 0 \\
0 & 0 & 0 & 0 & 0 & 10 & 0 & 0 & 0 & 0 \\
0 & 0 & 0 & 0 & 0 & 5 & 0 & 0 & 5 & 0 \\
0 & 0 & 0 & 0 & 1 & 6 & 0 & 0 & 0 & 3 \\
0 & 0 & 0 & 1 & 0 & 8 & 0 & 0 & 0 & 1 \\
0 & 0 & 0 & 0 & 0 & 0 & 0 & 0 & 10 & 0 \\
0 & 0 & 0 & 0 & 0 & 10 & 0 & 0 & 0 & 0 \\
		\end{array}
	\]
		\caption{Matriz de confusión para una $SNR=30dB$. \\
		El porcentaje de aciertos es paupérrimo $24\%$}
	\end{table}

	\begin{table}
	\[
		\begin{array}{cccccccccc}
10 & 0 & 0 & 0 & 0 & 0 & 0 & 0 & 0 & 0 \\
0 & 7 & 0 & 0 & 0 & 0 & 3 & 0 & 0 & 0 \\
0 & 0 & 9 & 0 & 0 & 0 & 0 & 1 & 0 & 0 \\
0 & 0 & 0 & 9 & 0 & 0 & 0 & 0 & 1 & 0 \\
1 & 0 & 0 & 0 & 6 & 0 & 0 & 0 & 0 & 3 \\
0 & 0 & 0 & 0 & 0 & 10 & 0 & 0 & 0 & 0 \\
0 & 3 & 0 & 0 & 0 & 0 & 7 & 0 & 0 & 0 \\
0 & 0 & 0 & 0 & 0 & 0 & 0 & 10 & 0 & 0 \\
0 & 0 & 0 & 4 & 0 & 0 & 0 & 0 & 6 & 0 \\
0 & 0 & 0 & 0 & 0 & 0 & 0 & 0 & 0 & 10 \\
		\end{array}
	\]
		\caption{Evaluación de la clasificación de la consonante. \\
		Matriz de confusión para una $SNR=20dB$ \\
		Porcentaje de aciertos de $84\%$}
	\end{table}
	
%ejemplo
%\section{Conclusiones}

Sometimos el algoritmo propuesto a las mismas señales de prueba pero
agregándoles distintos niveles de ruido obteniéndose las matrices de
confusión que se muestran en las tablas ….

En la prueba de SNR > 40 db, el algoritmo tuvo muy buenos resultados
para las señales de prueba a las que lo sometimos, obteniéndose
problemas para identificar los dígitos 2, 3, 4 y 6. En los primeros tres
dígitos se obtuvieron errores tanto en la detección de la vocal como
de la consonante a diferencia del caso del dígito 6 que solo falló
en la consonante. Creemos que el problema, en este caso, estuvo en la
pronunciación en las grabaciones correspondientes ya que escuchándolas
no se apreciaba el sonido de la J.

Dado que al aumentar el nivel de ruido la detección de la vocal
comete muchos errores, decidimos evaluar la detección de la consonante
proveyendo la vocal correspondiente en los casos de prueba. Los resultados
obtenidos fueron buenos con una SNR de 20 db, obteniéndose mayores
problemas al diferenciar  el dígito 4 (Vo) y el 8 (Bi). En el segundo
caso el problema se debe a que el sonido de la consonante del dígito
3 suele aproximarse al sonido del ruido blanco.

\section{Conclusión}

Si bien los primeros resultados de la clasificación eran excelentes,
la identificación de la vocal no es robusta respecto al ruido.
Esto se debió a que no siempre fue posible detectar todas las formantes,
en cuyo caso el algoritmo reemplazaba las formantes faltantes por formantes de orden superior.
Comparar estos vectores con la distancia euclídea es un error, ya que corresponden a características
distintas.

La clasificación de las consonantes se comportó mejor, siempre y cuando supiera la vocal correctamente.
Sin embargo, el bajo contenido energético de algunas consonantes limita la $SNR$ soportable





\nocite{*} \bibliographystyle{tfmpd} \bibliography{tfmpd}

\end{document}
